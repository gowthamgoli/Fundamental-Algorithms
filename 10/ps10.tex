%me=0 student solutions (ps file), me=1 - my solutions (sol file), me=2 - assignment (hw file)
\def\me{0}
\def\num{10}  %homework number
\def\due{Tuesday, December 1}  %due date
\def\course{CSCI-GA.1170-001/002 Fundamental Algorithms} %course name, changed only once
\def\name{GOWTHAM GOLI (N17656180)}   %student changes (instructor keeps!)
%
\iffalse
INSTRUCTIONS: replace # by the homework number.
(if this is not ps#.tex, use the right file name)

  Clip out the ********* INSERT HERE ********* bits below and insert
appropriate TeX code.  Once you are done with your file, run

  ``latex ps#.tex''

from a UNIX prompt.  If your LaTeX code is clean, the latex will exit
back to a prompt.  To see intermediate results, type

  ``xdvi ps#.dvi'' (from UNIX prompt)
  ``yap ps#.dvi'' (if using MikTex in Windows)

after compilation. Once you are done, run

  ``dvips ps#.dvi''

which should print your file to the nearest printer.  There will be
residual files called ps#.log, ps#.aux, and ps#.dvi.  All these can be
deleted, but do not delete ps1.tex. To generate postscript file ps#.ps,
run

  ``dvips -o ps#.ps ps#.dvi''

I assume you know how to print .ps files (``lpr -Pprinter ps#.ps'')
\fi
%
\documentclass[11pt]{article}
\usepackage{amsfonts}
\usepackage{latexsym}
\usepackage[lined,boxed,linesnumbered]{algorithm2e}
\usepackage{amsmath}
\usepackage{amsthm}
\usepackage{array}
\usepackage{amssymb}
\usepackage{amsthm}
\usepackage{epsfig}
\usepackage{psfrag}
\usepackage{color}
\usepackage{tikz}
\usepackage{enumerate}
\usetikzlibrary{calc,trees,positioning,arrows,fit,shapes,calc}
\usetikzlibrary{trees}
\usepackage{mathtools}
\usepackage{float}
\setlength{\oddsidemargin}{.0in}
\setlength{\evensidemargin}{.0in}
\setlength{\textwidth}{6.5in}
\setlength{\topmargin}{-0.4in}
\setlength{\textheight}{8.5in}

\newcommand{\handout}[5]{
   \renewcommand{\thepage}{#1, Page \arabic{page}}
   \noindent
   \begin{center}
   \framebox{
      \vbox{
    \hbox to 5.78in { {\bf \course} \hfill #2 }
       \vspace{4mm}
       \hbox to 5.78in { {\Large \hfill #5  \hfill} }
       \vspace{2mm}
       \hbox to 5.78in { {\it #3 \hfill #4} }
      }
   }
   \end{center}
   \vspace*{4mm}
}

\newcommand{\LCA}{\mbox{\sf LCA}}

\newcommand{\rs}{\rightsquigarrow}
\newcommand{\ls}{\leftsquigarrow}

\newcounter{pppp}
\newcommand{\prob}{\arabic{pppp}}  %problem number
\newcommand{\increase}{\addtocounter{pppp}{1}}  %problem number

%first argument desription, second number of points
\newcommand{\newproblem}[2]{
\ifnum\me=0
\ifnum\prob>0 \newpage \fi
\increase
\setcounter{page}{1}
\handout{\name, Homework \num, Problem \arabic{pppp}}{\today}{Name: \name}{Due:
\due}{Solutions to Problem \prob\ of Homework \num\ (#2)}
\else
\increase
\section*{Problem \num-\prob~(#1) \hfill {#2}}
\fi
}

%\newcommand{\newproblem}[2]{\increase
%\section*{Problem \num-\prob~(#1) \hfill {#2}}
%}

\def\squarebox#1{\hbox to #1{\hfill\vbox to #1{\vfill}}}
\def\qed{\hspace*{\fill}
        \vbox{\hrule\hbox{\vrule\squarebox{.667em}\vrule}\hrule}}
\newenvironment{solution}{\begin{trivlist}\item[]{\bf Solution:}}
                      {\qed \end{trivlist}}
\newenvironment{solsketch}{\begin{trivlist}\item[]{\bf Solution Sketch:}}
                      {\qed \end{trivlist}}
\newenvironment{code}{\begin{tabbing}
12345\=12345\=12345\=12345\=12345\=12345\=12345\=12345\= \kill }
{\end{tabbing}}

%%%%%\newcommand{\eqref}[1]{Equation~(\ref{eq:#1})}

\newcommand{\hint}[1]{({\bf Hint}: {#1})}
%Put more macros here, as needed.
\newcommand{\room}{\medskip\ni}
\newcommand{\brak}[1]{\langle #1 \rangle}
\newcommand{\bit}[1]{\{0,1\}^{#1}}
\newcommand{\zo}{\{0,1\}}
\newcommand{\C}{{\cal C}}

\newcommand{\nin}{\not\in}
\newcommand{\set}[1]{\{#1\}}
\renewcommand{\ni}{\noindent}
\renewcommand{\gets}{\leftarrow}
\renewcommand{\to}{\rightarrow}
\newcommand{\assign}{:=}
\newcommand{\cT}{\mathcal{T}}

\newcommand{\AND}{\wedge}
\newcommand{\OR}{\vee}

\newcommand{\Forr}{\mbox{\bf For }}
\newcommand{\To}{\mbox{\bf to }}
\newcommand{\Do}{\mbox{\bf Do }}
\newcommand{\Ifi}{\mbox{\bf If }}
\newcommand{\Then}{\mbox{\bf Then }}
\newcommand{\Elsee}{\mbox{\bf Else }}
\newcommand{\Whilee}{\mbox{\bf While }}
\newcommand{\Repeatt}{\mbox{\bf Repeat }}
\newcommand{\Until}{\mbox{\bf Until }}
\newcommand{\Returnn}{\mbox{\bf Return }}
\newcommand{\Swap}{\mbox{\bf Swap }}

\begin{document}

\ifnum\me=0
%\handout{PS\num}{\today}{Name: **** INSERT YOU NAME HERE ****}{Due:
%\due}{Solutions to Problem Set \num}
%
%I collaborated with *********** INSERT COLLABORATORS HERE (INDICATING
%SPECIFIC PROBLEMS) *************.
\fi
\ifnum\me=1
\handout{PS\num}{\today}{Name: Yevgeniy Dodis}{Due: \due}{Solution
{\em Sketches} to Problem Set \num}
\fi
\ifnum\me=2
\handout{PS\num}{\today}{Lecturer: Yevgeniy Dodis}{Due: \due}{Problem
Set \num}
\fi


\newproblem{Avoiding Double Infections}{8 points}

You are given a directed graph $G=(V,E)$ on $n$ nodes and $m$ edges,
where the node set $V = A\cup B$ consists of two disjoint subsets $A$
and $B$ of sizes $n_1$ and $n_2$ (so $n=n_1+n_2$). Nodes in $A$ are
``healthy'', while nodes in $B$ are ``infected''. Given a source $s\in
A$, your goal is to compute the shortest distance from $s$ to every
other healthy node which can pass through {\em at most one} infected
node (i.e., if the path from $s$ to $v$ contains at most one infected
$u$, this is OK, but if it contains two or more, this path is not
allowed when computing the shortest distance).

Define the following {\em directed} graph $G' = (V',E')$ on $2n_1+n_2$ nodes. The vertex
set of $V'$ of $G'$ is $V'= A_1\cup B \cup A_2$, where $A_1$ and $A_2$
are two copies of healthy nodes $A$. Two nodes in $A_1$ are connected
in $G'$ if and only if they are connected in $G$, and the same between
two nodes in $A_2$. The nodes in $B$ are more interesting. For every
original incoming edge $(a,b)\in E$, where $a\in A$ and $b\in B$, we
will put an edge $(a_1,b)$ in $E'$, where $a_1$ is the copy of $a$ in
$A_1$. Similarly, for every original outgoing edge $(b,a)\in E$, where
$a\in A$ and $b\in B$, we will put an edge $(b,a_2)$ in $E'$, where
$a_2$ is the copy of $a$ in $A_2$.

\begin{itemize}

\item[(a)](2 pts) Let $n',m'$ be the number of vertices and edges in
$G'$. Show that $n'\le 2n$ and $m'\le 2m$.

\ifnum\me<2
\begin{solution}

$n' = 2n_1 + n_2 = n+n_1 < n+n = 2n \implies n' < 2n$

Let number of edges within $A$ be $m_1$ and the number of edges within $B$ be $m_2$ and the number of edges between $A$ and $B$ be $m_3$. Hence $m = m_1+m_2+m_3$. 

$\therefore |E'| = m' = m_1 + m_1 + m_3 < m_1 + m_1 + m_3 + (m_2 + m_2 + m_3) = 2m \implies m' < 2m$
\end{solution}
\fi

\item[(b)](4 pts) Recall our original problem of computing the required
shortest distance in $G$ from $s$ to every other healthy node $a\in A$
which can pass through {\em at most one} infected node $b\in B$. Call
this distance $a[dis]$. Let $s_1$ and $s_2$ be the two copies of $s$
in $G'$. Using one ``appropriate'' BFS call on $G'$, show how to
compute the values $a[dis]$. Specifically, say what is the starting
node (call it $s'$) of your BFS call in $G'$. Also, after your BFS
call computed shortest distances $v'.d$ from $s'$ to $v'$, for every
$v'\in V'$, show how to compute the desired values $a[dis]$ for the
problem at hand (i.e., write an explicit formula for $a[dis]$ using
appropriate $v'.d$ values). Justify your algorithm.

\ifnum\me<2
\begin{solution}

Let $s'$ be $s_1$ i.e the copy of $s$ in $A_1$ and now call BFS on $s'$ in $G'$. This gives us the shortest distance from $s'$ to every other node in $G'$ i.e $v'.d$, for every $v' \in V'$.

However we are only interested in $v' \in A_1 \cup A_2$ i.e the copy of all the vertices of $A$. Consider some vertex $a \in A$. There can be multiple paths from $s$ to $a$ in $G$ i.e the paths containing zero infected nodes or one infected nodes or two infected nodes etc.

Now $a[dis]$ is the shortest distance from $s$ to $a$ containing atmost one infected node. Therefore this can be broken down into two parts i.e the minimum of the shortest distance from $s$ to $a$ containing 0 infected nodes, let it be $a[dis_0]$ or the shortest distance from $s$ to $a$ containing 1 infected nodes, let it be $a[dis_1]$.
\begin{equation*}
a[dis] = min\{a[dis_0], a[dis_1]\}
\end{equation*}

Let $a_1 \in A_1$ and $a_2 \in A_2$ be the copies of the vertex $a \in A$. Note that the shortest path from $s'$ to $a_1$ has 0 number of infected nodes in it as there are no edges from the $B$ to $A_1$ and the path from $s_2$ to $a_2$ has exactly 1 infected node as we need to visit exactly one node in $B$ to reach any node in $A_2$. Therefore substitute these values of $a_1.d = a[dis_0]$ and $a_2.d = a[dis_1]$ in the above equation
\begin{equation*}
a[dis] = min\{a_1.d,a_2.d\}
\end{equation*}
where $a_1$ and $a_2$ are the copies of $a$ in $A_1$ and $A_2$
\end{solution}
\fi

\item[(c)](2 pts) Show that the running time of your procedure is $O(m+n)$.

\ifnum\me<2
\begin{solution}

In the above algorithm we called BFS on $s'$ in $G'$. This step takes $O(|V'| + |E'|) = O(m + n)$ time and then we take minimum of $a_1.d$ and $a_2.d$ for every $a \in A$. This step takes $O(|A|) = O(|V|) = O(n)$ time. Therefore the total running time of the algorithm is $O(m+n)$
\end{solution}
\fi


\end{itemize}


\newproblem{Crazy Dogs}{15 points}

A fellow Moe and his buddy Joe live in a city $G=(V,E)$ which is an
undirected graph on $n$ vertices and $m$ edges, given in the adjacency
list form. Moe lives in a vertex $a$ and owns a crazy dog Mimi, while
Joe lives at a vertex $b$ and owns a crazy dog Kiki. This Sunday Moe
wants to take Mimi to a veterinarian clinic located at vertex $c$, while
Joe wants to take Kiki to the dance competition located at a vertex
$d$. One problem, though: the dogs hate each other, and if one of them
smells the other, all hell breaks loose. Luckily, they can smell each
other only if within distance at most $15$ in $G$, and you know that
$dist_G(a,b), dist_G(c,d)>15$. Moe and Joe would like to start at the
same time (with Moe and Mimi at $a$ and Joe and Kiki at $b$) and get
both dogs to their respective destinations $c$ and $d$ in the
smallest number of {\em steps} $t$. A {\em step} consists of both dogs
going to their respective neighboring vertices, or one dog going to a
neighboring vertex, and the other dog staying put, barking at the
pedestrians. Of course, such a step is possible only if the dogs stay
within distance $16$ or more both before and after the step.

Your job is to design an algorithm for Moe and Joe to compute $t$ (and
the optimal route), if the route exists, and analyze its running time.

\begin{itemize}

\item[(a)](5 pts) Using one or more runs of the BFS algorithm on $G$, fill in the
matrix $OK[x,y]$, where $OK[x,y]=1$, if it is OK for Mimi to be at
vertex $x$ and Kiki to be at vertex $y$ at the same time, and $OK[x,y]=0$
otherwise. How long did it take you to fill this matrix $OK[x,y]$?

\ifnum\me<2
\begin{solution}

For every node $x \in V$ compute $BFS(x)$\\ and $\forall y \in V$, if $\delta(x,y) > 15$ set $OK[x,y] = 1$ and 0 otherwise.

In the above algorithm BFS is called on every node of $G$. Therefore the running time is $O(n(m+n)) = O(mn)$
\end{solution}
\fi

\item[(b)](5 pts)  Design a graph $H=(V',E')$ whose vertex set consists of
possible ``ok configurations'' for Mimi and Kiki, and whose edge set
represents the possible single steps of your algorithm. Be sure to
formally define $V'$ and $E'$ as functions of $V$ and $E$ and the matrix
$OK$ from part (a). How long (in the worst case) did it take you to create
an adjacency list for $H$ (not counting what you did in part (a))?
What is the maximum $|V'|$ and $|E'|$?

\ifnum\me<2
\begin{solution}

If for any $x,y \in V$, $OK[x,y] = 1$ then we create a single node for both $x$ and $y$. Let this node be called $xy$. Therefore $V' = \{xy$ such that $OK[x,y]=1$ where $ x,y \in V\}$. Therefore creating $V'$ takes $O(n^2)$ time
\pagebreak

Let $x_1y_1$ and $x_2y_2$ be any two vertices in $V'$ then
\begin{itemize}
\item $(x_1y_1, x_2,y_2) \in E'$ if $OK[x_1,y_1] = 1$, $OK[x_2,y_2] = 1$ and $(x_1,x_2), (y_1,y_2) \in E$ i.e if Initially Mimi is at $x_1$ and Kiki is at $y_1$. Now each of them take one step, go to their respective neighbors $x_2$ and $y_2$ and they are still $>15$ units apart from each other
\item $(x_1y_1, x_2,y_2) \in E'$ if $OK[x_1,y_1] = 1$, $OK[x_1,y_2] = 1$ and $(y_1,y_2) \in E$ i.e if Initially Mimi is at $x_1$ and Kiki is at $y_1$. Now Mimi stays put at $x_1$ while Kiki goes to his neighboring vertex $y_2$ and they are still $>15$ units apart from each other
\item $(x_1y_1, x_2,y_1) \in E'$ if $OK[x_1,y_1] = 1$, $OK[x_2,y_1] = 1$ and $(x_1,x_2) \in E$ i.e if Initially Mimi is at $x_1$ and Kiki is at $y_1$. Now Kiki stays put at $y_1$ while Mimi to goes his neighboring vertex $x_2$ and they are still $>15$ units apart from each other
\end{itemize}

Let $xy \in V'$. Now we need to analyze the number of outgoing edges from $xy$. Let $m_x$ be the number of outgoing edges from $x$ to it's neighboring vertices $x_1, \ldots, x_{m_x}$ and $m_y$ be the number of outgoing edges from  $y$ to it's neighboring vertices $y_1, \ldots, y_{m_y}$ in $G$. Now for each $x_i \in \{x_1, \ldots, x_{m_x}\}$ and $y_j \in \{y_1, \ldots, y_{m_y}\}$, we check if $OK[x_i,y_j]$ is 1 and add an edge from $xy$ to $x_iy_j$. Therefore the total time taken to add the number of outgoing edges from $xy$ is $m_xm_y$.  

Let $x_1,\ldots,x_n \in V$ and $y_1,\ldots,y_n \in V$ be all the vertices of $G$. In the worst case if $OK[x_1,y_i] = 1$ for each $i = 1, \ldots, n$ then the time taken to add all the outgoing edges from each vertex $x_1y_i$ is $m_{x_1}(m_{y_1} + \ldots + m_{y_n}) = m_{x_1}m$. Similarly we can analyse for all $x_1, \ldots, x_n \in V$, Therefore the total time taken to create $E'$ is $m(m_{x_1} + \ldots + m_{x_n}) = m^2$

Therefore the total time to create the adjacency list of $H$ is the time to create $V'$ and the time taken to $E'$ = $O(m^2+n^2)$.

In the worst case when all the vertices are more than 15 units from each other the entire $OK$ matrix will be 1. Therefore $|V|_{max} = n^2$ and using the above analysis of how outgoing edges are created it is easy to see that $|E|_{max} = m^2$
\end{solution}
\fi

\item[(c)](5 pts) Describe the original problem as a shortest path computation
on $H$. Finally, solve the original problem, and help Moe and
Joe. Analyze the overall running time of your algorithm, as a function
of $n$ and $m$.

\ifnum\me<2
\begin{solution}

All the vertices and edges of $H$ are of $OK$ configurations. So simply call $BFS$ on $ab$ and get the shortest distance from $ab$ to $cd$ and return $\delta(ab,cd)$. Running time will be $O(|V'|+|E'|) = O(m^2+n^2)$
\end{solution}
\fi

\end{itemize}



\newproblem{Diameter of Tree}{12 points}

The diameter of an undirected tree $T=(V,E)$ on $n$ vertices $V$
(and $(n-1)$ edges $E$) is the largest of all shortest paths distances
in the tree: $D = \max_{x,y\in V} \delta(x,y)$. You will design an $O(n)$
algorithm to compute $D$ and will prove its correctness as follow. 

\begin{itemize}
\item[(a)](7 pts) Let $r$ be the root of $T$. Let $b$ is the furthest node from $r$ 
in $T$. Show that the diameter path in $T$ either ends or starts at $b$.

\ifnum\me<2
\begin{solution}

Let $\delta(a,b)$ be the diameter of $T$ and $b$ is the farthest node from $r$.  Note that the root $r$ may or may not lie in the path between $a$ and $b$.

Consider that $r$ lies in the path between $a$ and $b$ i.e $r$ is the lowest common ancestor of $a$ and $b$. Therefore $\delta(a,b) = \delta(a,r) + \delta(r,b)$. Suppose that $\delta(a,b)$ is not the diameter of $T$ and let $\delta(a,c)$ be the diameter of $T$.  Therefore $\delta(a,c) = \delta(a,r) + \delta(r,c)$. We know that $\delta(r,c) < \delta(r,b)$ as $b$ is the farthest node from $r$. Hence $\delta(a,c) < \delta(a,b)$. Hence our assumption is wrong i.e $c$ has to be the farthest node from $r$. Hence $c = b$ 

Now consider that $r$ doesn't lie in the path between $a$ and $b$ and let $p$ be the lower common ancestor of $a$ and $b$. Therefore $\delta(a,b) = \delta(a,p) + \delta(p,b)$. Suppose that $\delta(a,b)$ is not the diameter of $T$ and let $\delta(a,c)$ be the diameter of $T$. Therefore $\delta(a,c) = \delta(a,p) + \delta(p,c)$. Since $b$ is the farthest node from $r$ we have, $\delta(r,c) < \delta(r,b) \implies \delta(r,p) + \delta(p,c) < \delta(r,p) + \delta(p,b) \implies \delta(p,c) < \delta(p,b)$. Hence $\delta(a,c) < \delta(a,b)$.  Hence our assumption is wrong i.e $c$ has to be the farthest node from $r$. Hence $c = b$ 

Note that in the above two cases we considered that the diameter ends at $b$. We can similarly argue when the diameter starts at $b$. Therefore we can conclude that the diameter path in $T$ starts or ends at $b$ where $b$ is the farthest node from the root of the tree
\end{solution}
\fi

\item[(b)] (5 pts) Assuming part (a), irrespective of whether or not you solved it, 
design an $O(n)$ algorithm to compute $D$. For partial credit, give a slower algorithm.

\ifnum\me<2
\begin{solution}

\begin{itemize}
\item Pick the root node $r$ and perform BFS on it
\item Let $b$ be the farthest node from $r$. Therefore from part (a), the diameter of $T$ either starts or ends at $b$
\item Perform BFS on $b$. Let $a$ be the the farthest node from $b$
\item $\delta(a,b)$ is the the diameter of $T$
\end{itemize}
In the above algorithm, BFS is called twice so the running time is $O(m+n)$ but we know that $ m = n-1$. Therefore the running time is $O(n)$
\end{solution}
\fi

\end{itemize}

\newproblem{Linear Time Topological Sort}{6 points}
Give an algorithm that determines whether or not given a undirected graph $G=(V,E)$ contains a cycle. Your algorithm should run in $O(V)$ time, independent of $|E|$. Make sure you prove the correctness of your algorithm.


\ifnum\me<2
\begin{solution}

\subsection*{Algorithm}

There will be a cycle in $G$ if in the DFS of $G$, any unexplored edge visits a node that has already been visited earlier i.e this unexplored edge is a backward edge in the DFS forest of $G$. Therefore if the DFS yields no back edges i.e it contains only tree edges then $G$ is acyclic or else $G$ contains a cycle

\subsection*{Correctness and Running Time}

If the given graph $G$ is acyclic then the DFS forest will not contain any egde $(u,v)$ such that $v$ is an ancestor of $u$. Suppose there exist such an edge $(u,v)$ then $DFS(v)$ will visit $u$ as $v$ is the ancestor of $u$ and then $DFS(u)$ visits $v$ which is already visited thus resulting in a cycle. Therefore it contradicts our assumption and hence the edge $(u,v)$ can't exist i.e DFS yields no back edges and contains only tree edges
if $G$ is acyclic

Therefore if $G$ is acyclic, then maximum number of tree edges a graph can have is at most $|V|-1$. Therefore a single run of DFS is sufficient to check for back edges in this case. Therefore the running time will be $O(m+n) = O(2n) = O(n)$

Suppose that $G$ contains a cycle, then DFS of $G$ will contain at least one back edge. Note that while running DFS this back edge can be found before seeing $|V|$ edges. Therefore there are only  $O(|V|)$ number of operations and the back edge will be found before that. Hence the running time will be $O(n)$.
Therefore the algorithm will run in $O(|V|)$, independent of $|E|$

\end{solution}
\fi
\end{document}


