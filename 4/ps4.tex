%me=0 student solutions (ps file), me=1 - my solutions (sol file), me=2 - assignment (hw file)
\def\me{0}
\def\num{4}  %homework number
\def\due{Tuesday, October 6}  %due date
\def\course{CSCI-GA.1170-001/002 Fundamental Algorithms} %course name, changed only once
\def\name{GOWTHAM GOLI (N17656180)}   %student changes (instructor keeps!)
%
\iffalse
INSTRUCTIONS: replace # by the homework number.
(if this is not ps#.tex, use the right file name)

  Clip out the ********* INSERT HERE ********* bits below and insert
appropriate TeX code.  Once you are done with your file, run

  ``latex ps#.tex''

from a UNIX prompt.  If your LaTeX code is clean, the latex will exit
back to a prompt.  To see intermediate results, type

  ``xdvi ps#.dvi'' (from UNIX prompt)
  ``yap ps#.dvi'' (if using MikTex in Windows)

after compilation. Once you are done, run

  ``dvips ps#.dvi''

which should print your file to the nearest printer.  There will be
residual files called ps#.log, ps#.aux, and ps#.dvi.  All these can be
deleted, but do not delete ps1.tex. To generate postscript file ps#.ps,
run

  ``dvips -o ps#.ps ps#.dvi''

I assume you know how to print .ps files (``lpr -Pprinter ps#.ps'')
\fi
%
\documentclass[11pt]{article}
\usepackage{amsfonts}
\usepackage{latexsym}
\usepackage[lined,boxed,linesnumbered]{algorithm2e}
\usepackage{amsmath}
\usepackage{amsthm}
\usepackage{array}
\usepackage{amssymb}
\usepackage{amsthm}
\usepackage{epsfig}
\usepackage{psfrag}
\usepackage{color}
\usepackage{tikz}
\usetikzlibrary{calc,trees,positioning,arrows,fit,shapes,calc}
\usetikzlibrary{trees}
\usepackage{mathtools}
\usepackage{float}
\setlength{\oddsidemargin}{.0in}
\setlength{\evensidemargin}{.0in}
\setlength{\textwidth}{6.5in}
\setlength{\topmargin}{-0.4in}
\setlength{\textheight}{8.5in}
\newtheorem{theorem}{Theorem}
\newcommand{\handout}[5]{
   \renewcommand{\thepage}{#1, Page \arabic{page}}
   \noindent
   \begin{center}
   \framebox{
      \vbox{
    \hbox to 5.78in { {\bf \course} \hfill #2 }
       \vspace{4mm}
       \hbox to 5.78in { {\Large \hfill #5  \hfill} }
       \vspace{2mm}
       \hbox to 5.78in { {\it #3 \hfill #4} }
      }
   }
   \end{center}
   \vspace*{4mm}
}
\definecolor{myblue}{RGB}{56,94,141}
\newcounter{pppp}
\newcommand{\prob}{\arabic{pppp}}  %problem number
\newcommand{\increase}{\addtocounter{pppp}{1}}  %problem number

%first argument desription, second number of points
\newcommand{\newproblem}[2]{
\ifnum\me=0
\ifnum\prob>0 \newpage \fi
\increase
\setcounter{page}{1}
\handout{\name, Homework \num, Problem \arabic{pppp}}{\today}{Name: \name}{Due:
\due}{Solutions to Problem \prob\ of Homework \num\ (#2)}
\else
\increase
\section*{Problem \num-\prob~(#1) \hfill {#2}}
\fi
}

%\newcommand{\newproblem}[2]{\increase
%\section*{Problem \num-\prob~(#1) \hfill {#2}}
%}

\def\squarebox#1{\hbox to #1{\hfill\vbox to #1{\vfill}}}
\def\qed{\hspace*{\fill}
        \vbox{\hrule\hbox{\vrule\squarebox{.667em}\vrule}\hrule}}
\newenvironment{solution}{\begin{trivlist}\item[]{\bf Solution:}}
                      {\qed \end{trivlist}}
\newenvironment{solsketch}{\begin{trivlist}\item[]{\bf Solution Sketch:}}
                      {\qed \end{trivlist}}
\newenvironment{code}{\begin{tabbing}
12345\=12345\=12345\=12345\=12345\=12345\=12345\=12345\= \kill }
{\end{tabbing}}

%\newcommand{\eqref}[1]{Equation~(\ref{eq:#1})}

\newcommand{\hint}[1]{({\bf Hint}: {#1})}
%Put more macros here, as needed.
\newcommand{\room}{\medskip\ni}
\newcommand{\brak}[1]{\langle #1 \rangle}
\newcommand{\bit}[1]{\{0,1\}^{#1}}
\newcommand{\zo}{\{0,1\}}
\newcommand{\C}{{\cal C}}

\newcommand{\nin}{\not\in}
\newcommand{\set}[1]{\{#1\}}
\renewcommand{\ni}{\noindent}
\renewcommand{\gets}{\leftarrow}
\renewcommand{\to}{\rightarrow}
\newcommand{\assign}{:=}

\newcommand{\AND}{\wedge}
\newcommand{\OR}{\vee}

\newcommand{\Forr}{\mbox{\bf For }}
\newcommand{\To}{\mbox{\bf to }}
\newcommand{\Do}{\mbox{\bf Do }}
\newcommand{\Ifi}{\mbox{\bf If }}
\newcommand{\Thenn}{\mbox{\bf Then }}
\newcommand{\Elsee}{\mbox{\bf Else }}
\newcommand{\Whilee}{\mbox{\bf While }}
\newcommand{\Repeatt}{\mbox{\bf Repeat }}
\newcommand{\Until}{\mbox{\bf Until }}
\newcommand{\Returnn}{\mbox{\bf Return }}
\newcommand{\Swap}{\mbox{\bf Swap }}

\begin{document}

\ifnum\me=0
%\handout{PS\num}{\today}{Name: **** INSERT YOU NAME HERE ****}{Due:
%\due}{Solutions to Problem Set \num}
%
%I collaborated with *********** INSERT COLLABORATORS HERE (INDICATING
%SPECIFIC PROBLEMS) *************.
\fi
\ifnum\me=1
\handout{PS\num}{\today}{Name: Yevgeniy Dodis}{Due: \due}{Solution
{\em Sketches} to Problem Set \num}
\fi
\ifnum\me=2
\handout{PS\num}{\today}{Lecturer: Yevgeniy Dodis}{Due: \due}{Problem
Set \num}
\fi





\newproblem{Fast $k$-way Merging}{10 points}

Using a min-heap in a clever way, give $O(n\log k)$-time algorithm to
merge $k$ sorted arrays $A_1\ldots A_k$ of size $n/k$ each into one
sorted array $B$.  Write the pseudocode of your algorithm using
procedures {\sc Build-Heap}, {\sc Extract-Min} and {\sc Insert}.

\ifnum\me<2
\begin{solution}

\section*{Algorithm}
\begin{itemize}
\item Build a Heap using the first elements of the given arrays $A_1, \ldots, A_k$
\item Extract Minimum from the heap and insert into $B$
\item Pick the next element from the array, the popped element of the heap came from and insert into this heap
\item Keep repeating until all the elements of the given arrays are inserted into the heap
\end{itemize}

\section*{Psuedocode}

\IncMargin{1em}
\begin{algorithm}[H]
\TitleOfAlgo{{\sc MergeHeaps}($A_1,\ldots,A_k, B$)}
$Heap$ $\leftarrow$ {\sc Build-Heap}($A_1[0],\ldots,A_k[0]$)\\
$B \leftarrow$ {\sc newArray}(n)\\
$c$ $\leftarrow$ 0\\
\While{Heap is not empty}{
	$A_i[p]$ $\leftarrow$ = {\sc Extract-Min}($Heap$)\\
	$B[c]$ $\leftarrow$ $A_i[p]$\\
	\If{p $<$ n/k}{
		{\sc Insert}$(A[p+1], Heap)$\\
	}
	$c$$++$
}
\Returnn $B$
\caption{Merge sorted arrays $A_1,\ldots,A_k$ using a min-heap}
\end{algorithm}
\pagebreak
\section*{Time Complexity}
In the first step, time taken to build heap of size $k$ is $O(k\log k)$. 
Time taken to extract minimum from the heap and then insert into the heap is $O(\log k)$ and this is done $n$ times, therefore $O(n\log k)$
Therefore, total time takes is $O(k\log k) + O(n\log k) = O(n\log k)$

\end{solution}
\fi



\newproblem{Finding a Repetitor}{26 (+14) Points}

Your are given an array $A[1]\ldots A[n]$ of $n$ ``objects''. You
have a magic unit-time procedure $Equal(A[i],A[j])$, which will tell
if objects $A[i]$ and $A[j]$ are the same.
%(This test is naturally assumed to be commutative and associative,
%so no surprises there.)
Unfortunately, there is no other way to get any meaningful
information about the objects: e.g., cannot ask if $A[i]$ is
``greater'' than $A[j]$ of if it is more ``sexy'', etc., just the
equality test.  We say that $A$ is a {\em repetitor} if it contains
strictly more than $n/2$ elements which are all pairwise the same.
In this case any of $A$'s (at least $n/2$) repetitive elements is
called {\em dull}. For example, if the ``object'' is a string, the
array $(boring, funny, cute, boring, boring)$ is a repetitor where
$boring$ is dull while $funny$ is not. On the other hand, the array
$(hello, hi, bonjorno, hola, whasup)$ is not a repetitor.  You goal
is to determine if $A$ is a repetitor, and, if so, output its dull
``object'' (which is clearly unique).

\begin{itemize}

\item[(a)] (8 points) Design a simple divide-and-conquer algorithm for this
problem running in time $O(n\log n)$. Make sure you argue the
correctness and the running time.\\
\hint{Prove that if $A$ is a repetitor, at least one of its
``halves'' is as well.}

\ifnum\me<2
\begin{solution}
\begin{theorem}
If $A$ is a repetitor, at least one of its
``halves'' is as well
\end{theorem}
\begin{proof}

Let $p$ be the dull of $A$. 
If the left half of $A$ is not a repetitor then there are less than $n/4$ occurrences of $p$  in $A[1 \ldots mid]$ which implies that the right half $A[mid+1 \ldots n]$ has to have more than $n/4$ occurrences of $p$ for $p$ to be a dull of $A$. Therefore, the right half is a repetitor. Similarly, we can argue for the left half to be a repetitor. 

If neither of the halves of $A$ is a repetitor, it means that there are less than $n/4$ occurrences of $p$  in $A[1 \ldots mid]$ and  $A[mid+1 \ldots n]$. Therefore, $p$ cannot be the dull in this case

If both the halves of $A$ are repetitors then $A[1 \ldots mid]$ has to have more than $n/4$ occurrences of $p$ and  $A[mid+1 \ldots n]$ has to have more than $n/4$ occurrences of $p$ for $p$ to be a dull of $A$.

Therefore we can conclude that If $A$ is a repetitor, at least one of its
``halves'' is as well 
\end{proof}
Using the above theorem, we can implement a $O(n \log n)$ algorithm as follows

\section*{Psuedocode}
\IncMargin{1em}
\begin{algorithm}[H]
\TitleOfAlgo{{\sc FindDull}(A, low, high)}
\If{low is high}{
\textbf{return} $A[low]$
}
\Else{
	$mid$ $\leftarrow$ (low + high)/2\\
	$leftDull$ $\leftarrow$ {\sc FindDull}($A, low, mid$)\\
	$rightDull$ $\leftarrow$ {\sc FindDull}($A, mid+1, high$)\\
	\If{leftDull is rightDull}{
		\Returnn $leftDull$
	}
	
	\Else{
		$lDullCount$ = {\sc CountOccurrences}($leftDull, A[low \ldots high$])\\
		$rDullCOunt$ = {\sc CountOccurrences}($rightDull, A[low \ldots high$])\\
		
		\uIf{lDullCount $>$ (high - low)/2}{
			\Returnn$ leftDull$
		}
		\uElseIf{rDullCount $>$ (high - low)/2}{
			\Returnn $rightDull$
		}	
		\uElse{
			\Returnn {\sc NoDull}
		}
	}
}

\caption{Alorithm to find the dull OF $A$ in $O(n \log n)$ time}
\end{algorithm}

\end{solution}
\fi

\item[(b)] (4 Points) Remember, if $A$ was an integer array, the
procedure {\sc Partition}$(A,p,r)$ (see Section 7.1) makes $x=A[r]$
the pivot element and returns the index $q$, where the new value of
$A[q]$ contains the pivot $x$, the new values $A[p\ldots q-1]$
contain elements less or equal to $x$, and the new values
$A[q+1\ldots r]$ contain values greater than $x$. Write the
pseudocode of the modified procedure {\sc New-Partition}$(A,p,r)$,
which only uses the $Equal$ operator and returns $q$ such that
$A[q+1\ldots r]$ contain all the elements equal to $x$ (while
$A[p\ldots q]$ contain all other elements).

\ifnum\me<2
\begin{solution}
\section*{Psuedocode}
\IncMargin{1em}
\begin{algorithm}[H]
\TitleOfAlgo{{\sc New-Partition}$(A,p,r)$}
$x \leftarrow A[r]$\\
$i \leftarrow p-1$\\
\For{j = p to r-1}{
	\If{A[j] $\neq$ x}{
		$i \leftarrow i+1$\\
		{\sc Swap}($A[i], A[j])$\\
	}
}
{\sc Swap}($A[i+1], A[r])$\\
\Returnn $i+1$
\caption{Modified procedure {\sc New-Partition}$(A,p,r)$ }
\end{algorithm}

\end{solution}
\fi

\item[(c)] (2 points) Consider the following, more general, algorithm
{\sc Repeat}$(A,n,t)$, which tells if some element of $A[1]\ldots
A[n]$ is repeated at least $t$ times. (Clearly, {\sc Repetitor} can
just call {\sc Repeat} with $t=n/2+1$.)

\begin{code}
{\sc Repeat}$(A,~n,~t)$\\
\> \Ifi $n<t$ \Returnn $no$\\
\> Pick $i\in \{1\ldots n\}$ {\em at random}.\\
\> $Swap(A[i],A[n])$\\
\> $q\leftarrow${\sc New-Partition}$(A,~1,~n)$\\
\> \Ifi $n-q\ge t$ \Thenn \Returnn $(yes,A[n])$ \\
\> \Returnn {\sc Repeat}$(A,~q,~t)$
\end{code}

Argue that the algorithm above is correct.

\ifnum\me<2
\begin{solution}

It is quite obvious that $t \leq n$, Therefore if $t > n$ the algorithm returns \textit{no}

Let the randomly picked element be $key$. {\sc New-Partition}$(A, 1, n)$ returns $A$ such that $A[1 \ldots q-1] \neq key$ and $A[q \ldots n] = key$. Therefore the array has $n-q$ occurrences of $key$

If $n-q \geq t$ then $key$ is repeated atleast $t$ times. Else we keep repeating the procedure with a new key each time until we find an element that is repeated atleast t times
\end{solution}
\fi

\item[(d)] (3 points) Argue that the algorithm above
always terminates in time $O(n^2)$ (irrespective of the random
choices of $i$).

\ifnum\me<2
\begin{solution}   ***************** INSERT YOUR SOLUTION HERE ***************   \end{solution}
\fi

\item[(e)] (3 points) Give an example of an (integer) array $A$ and a
value $t\ge 2$ where the algorithm indeed takes time $\Omega(n^2)$.

\ifnum\me<2
\begin{solution}   ***************** INSERT YOUR SOLUTION HERE ***************   \end{solution}
\fi

\item[(f)] (4 Points) Let $T(n)$ be the worst case (over all
arrays $A[1\ldots n]$ and $t>n/2$ such that $A$ contains $t$
identical elements) of the {\em expected} running time of {\sc
Repeat}$(A,n,t)$ (over the random choice of $i$). For concreteness,
assume {\sc New-Partition} takes time exactly $n$. Prove that
$$T(n) \le \frac{1}{2}\cdot T(n-1) + n$$
\hint{Prove that in this case no recursive sub-call will be made
with probability $t/n > 1/2$.}

\ifnum\me<2
\begin{solution}
\end{solution}
\fi

\item[(g)] (2 points) Show by induction that $T(n)\le 2n$.

\ifnum\me<2
\begin{solution}   ***************** INSERT YOUR SOLUTION HERE ***************   \end{solution}
\fi

\item[(h$^{*}$)] ({\bf Extra Credit}; 6 points)
Consider the following test for repetitor. For 100 times, run {\sc
Repeat}$(A,n,n/2+1)$ for at most $4n$ steps. If one of these 100
runs ever finishes within $4n$ steps, use that answer. If none of
the $100$ runs terminates within $4n$ steps, return $no$. Argue that
the running time of this procedure is $O(n)$. Then argue that the
probability it returns the incorrect $no$ answer (when it should
have returned $yes$)
is at most $2^{-100}$.\\
\hint{Show that when the answer is $yes$, the probability of not
finding this answer in $4n$ steps is at most $1/2$. Google for
``Markov's inequality'' if you want to be formal.}

\ifnum\me<2
\begin{solution}   ***************** INSERT YOUR SOLUTION HERE ***************   \end{solution}
\fi

\item[(i$^{**}$)] ({\bf Extra Credit}; 8 points)
Try to design $O(n)$ deterministic test for a repetitor.

\ifnum\me<2
\begin{solution}   ***************** INSERT YOUR SOLUTION HERE ***************   \end{solution}
\fi

\end{itemize}



\newproblem{Quicksort}{12 points}


Assume we are given an array $A[1\ldots n]$ of $n$ {\em distinct}
integers and that $n=2k$ is {\em even}.

\begin{itemize}

\item[(a)] (4 points) Let $pivot(A)$ denote the rank of the pivot
element at the end of the partition procedure, and assume that we
choose a random element $A[i]$ as a pivot, so that $pivot(A) = i$ with
probability $1/n$, for all $i$. Let $smallest(A)$ be the length of the
smaller sub-array in the two recursive subcalls of the {\sc
Quicksort}.  Notice, $smallest(A) = \min(pivot(A)-1,n-pivot(A))$ and
belongs to $\set{0\ldots k-1}$, since $n=2k$ is even. Given $0\le j\le
k-1$, what is the probability that $smallest(A)=j$?

\ifnum\me<2
\begin{solution}
Given that the probability to choose a random element $A[i]$ as a pivot is $1/n$. 

\end{solution}
\fi

\item[(b)] (3 points) Compute the {\em expected value} of $smallest(A)$;
i.e., $\sum_{j=0}^{k-1} \Pr(smallest(A)=j)\cdot j$.\\
\hint{If you solve part (a) correctly, no big computation is needed
here.}

\ifnum\me<2
\begin{solution}   ***************** INSERT YOUR SOLUTION HERE ***************   \end{solution}
\fi

\item[(c)] (5 points) Write a recurrence equation for the running time $T(n)$
of {\sc Quicksort}, assuming that at every level of the recursion the
corresponding sub-arrays of $A$ are partitioned {\em exactly} in the
ratio you computed in part (b). Solve the resulting recurrence
equation. Is it still as good as the average case of randomized {\sc
Quicksort}?

\ifnum\me<2
\begin{solution}   ***************** INSERT YOUR SOLUTION HERE ***************   \end{solution}
\fi

\end{itemize}



\newproblem{The $k$ Smallest Elements}{12 points}



We wish to implement a data structure $D$ that maintains the $k$
smallest elements of an array $A$. The data structure should allow the
following procedures:
\begin{itemize}

\item $D\leftarrow$ {\sc Initialize}($A,n,k$) that initializes $D$ for a
  given array $A$ of $n$ elements.

\item {\sc Traverse}($D$), that returns the $k$ smallest elements of
  $A$ {\em in sorted order}.

\item {\sc Insert}($D, x$), that updates $D$ when an element $x$ is
  inserted in the array $A$.

\end{itemize}

\noindent We can implement $D$ using one of the following data
structures: (i) an unsorted array of size $k$; (ii) a sorted array of
size $k$; (iii) a max-heap of size $k$.

\begin{itemize}

\item[(a)] (4 points) For each of the choices (i)-(iii), show that the
  {\sc Initialize} procedure can be performed in time $O(n + k \log
  n)$.

\ifnum\me<2
\begin{solution}   ***************** INSERT YOUR SOLUTION HERE ***************   \end{solution}
\fi

\item[(b)] (3 points) For each of the choices (i)-(iii), compute the
  best running time for the {\sc Traverse} procedure you can think
  of. (In particular, tell your procedure.)

\ifnum\me<2
\begin{solution}   ***************** INSERT YOUR SOLUTION HERE ***************   \end{solution}
\fi

\item[(c)] (5 points) For each of the choices (i)-(iii), compute the
  best running time for the {\sc Insert} procedure you can think
  of. (In particular, tell your procedure.)

\ifnum\me<2
\begin{solution}   ***************** INSERT YOUR SOLUTION HERE ***************   \end{solution}
\fi

\end{itemize}




\end{document}


